\section{Question 1}
The default implementation of the branch prediction simulation has the
following low-hanging fruits, in terms of speed optimizations.
\begin{enumerate}
  \item The counters for the branch predictor's performance are updated in two
  nested \texttt{if} statements, which means two branches. Since the lab is all
  about branch prediction, I believe this is the problem that we were supposed
  to notice. I replaced the individual counter variables with a 2 by 2 array
  and rewrote the updating logic so the compiler can use \texttt{cmov}
  instructions.
  \item The predictor has a base class which declares both
  \texttt{makePrediction} and \texttt{makeUpdate} as virtual functions. This
  causes indirect jumps, which stress the BTB, and is completely unnecessary,
  as the BranchPredictor implementation that will be used is known at
  compile-time. I fixed this by removing the base BranchPredictor class
  completely. I don't need it to implement multiple predictors, because I don't
  need to use them via generic pointers.
\end{enumerate} 

The implementation also has stylistic issues - pretty much all C++ style guides
I've read agree that class names should start with an uppercase letter
(see \texttt{myBranchPredictor}), and variable names should start with
a lowercase letter or underscore (see \texttt{BP}).

Fixing the above issues did not yield a signficantly faster running time for
the tests on \texttt{linerva}. So I might have missed the real issue.

\section{Question 2}
I chose to submit the Alpha 21264 branch predictor, outlined in Slide 24 in the
class lecture notes. The predictor is a tournament predictor, choosing between
a local history-based predictor, and a global history-based predictor. The
choice is made using a table of saturating 2-bit counters, based on global
history. My choice is based on the trade-off between accuracy and implementation
time.

I used the exact choices in the slide:
\begin{itemize}
  \item The local history-based predictor uses a 10kb local history table
  (1,024 entries with 10 history bits per entry), plus a 3kb local prediction
  table ($2^{10} = 1,024$ 3-bit saturating counters).
  \item The global history predictor uses a 12-bit global history register and
  a 8kb table ($2^{12} = 4,096$ 2-bit saturating counters).
  \item The predictor selector also uses a 8kb table ($2^{12} = 4,096$ 2-bit
  saturating counters).
\end{itemize}

These choices used up 29 kilobits out of the 33 kilobit budget. I was tempted
to use the remaining 4 kilobits to implement a loop predictor, but I gave up
when I realized I would have to figure out a way to integrate it with the other
predictors. The predictor's performance is shown in table
\ref{q2:predictor_acc}.

\begin{table}[htb]
\center
\begin{tabular}{lrrr}
\hline
Test & Total Accuracy & Taken Accuracy & Not Taken Accuracy \\
\hline
applu & 98.21\% & 96.00\% & 98.77\%  \\
\hline
art & 98.98\% & 99.31\% & 97.86\%  \\
\hline
bzip2 & 98.11\% & 98.40\% & 95.17\%  \\
\hline
cc1 & 92.01\% & 92.21\% & 91.79\%  \\
\hline
crafty & 91.33\% & 90.84\% & 91.70\%  \\
\hline
equake & 97.98\% & 97.40\% & 98.81\%  \\
\hline
gap & 96.09\% & 96.36\% & 95.73\%  \\
\hline
gzip & 94.08\% & 95.24\% & 92.37\%  \\
\hline
mesa & 99.81\% & 99.78\% & 99.84\%  \\
\hline
parser & 94.02\% & 95.44\% & 90.98\%  \\
\hline
swim & 99.73\% & 86.06\% & 99.79\%  \\
\hline
{}Average & 96.40\% & 95.19\% & 95.71\%  \\
\hline
{}Total & 95.80\% & 96.67\% & 94.06\%  \\
\hline
\end{tabular}

\caption{The accuracy of the submitted branch predictor (Alpha 21264) on the
SPEC 2000 benchmark suite. }
\label{q2:predictor_acc}
\end{table}

I did some Google-based research, and looked closely at the result of the 2004
Championship Branch Prediction. I also skimmed through the 2006 2nd
Championship Branch Prediction. The winning predictors were very complex, and
didn't promise a 98\% accuracy, which was rather discouraging. The conclusion
was sad, because I broke up my implementation into nice, reusable building
blocks using C++ templates and operator overloading.

I tried implementng a perceptron-based predictor, and it yielded weak results
(around 85\% accuracy), so I gave up on that avenue as well.

\section{Question 3}
I implemented the Alpha 21264 predictor, exactly as presented in class.
Therefore, I can claim with 100\% certainty that the predictor can be
implemented in hardware :)

\section{Question 4}
Every branch predictor will have to deal with the fact that several branch
predictions will be in flight at the same time, because branch resolution takes
a long time (hence the need for a predictor), and the instruction stream might
contain branching instructions that are close together, and latter branches must
be pushed through the pipeline before the first branch is resolved.

The first consequence of this problem is that the real predictor accuracy will
be lower than the accuracy given by our simulator, because we will not be able
to update the predictor's state.

The second consequence is that we'll have to update crucial predictor state
(e.g., the global history register) speculatively, in order to get a decent
prediction accuracy. The global history is a single register, so it's
reasonable to make it a part of the pipeline. On the other hand, in order to
update local histories speculatively, we'll need to build a fast buffer of
uncomitted updates -- this means a small fully-associative cache that can hold
many versions of the same local history table entry, together with logic for
committing and killing entries.
